% !TeX root = ../main.tex

\chapter{绪论}

在大数据时代,随着数据量的迅速增加,我们往往需要更低的时间复杂度算法来完成任务。此外,
在一些实际场景中,我们的数据是动态的,即数据集合会随着时间的推移而发生变化。显然,如果数据量极其庞大,
当训练数据动态变化时,重新训练整个模型的代价是不可行的。

为了解决这些问题,一个很自然的想法是构建一个小规模的训练集,以便我们可以在小规模集合而不是所有数据上运行现有的算法。
coreset,最初出现于计算几何研究中的概念,已成为许多大规模机器学习问题中广泛使用的数据压缩技术。作为一种简洁的
数据压缩技术,coreset还具有一些优秀的性质。例如,coreset通常是可组合的,因此可以应用于分布式计算环境之中。
此外,它往往能在流算法和包含数据增删的动态算法中取得较好的表现。

然而,现有的coreset算法仍然有较大的改进空间。一个主要的瓶颈是,多数coreset算法容易受到异常点(离群点)的影响
而导致性能下降,而现实数据集中往往有许多噪声。许多现有研究也表明,只需在训练样本中加入少数攻击点,就可能导致原模型的效果显著下降。
因此,如何设计一个鲁棒性更好的coreset算法是一个重要的研究方向。

为了简要说明现有的coreset方法对离群值的敏感性,我们可以以一种基于采样的coreset算法为例。这个算法需要对每个数据点计算一个“敏感度,
该敏感度衡量了该数据项对整个数据集的重要程度;然而,它往往会给远离数据多数部分的点分配高敏感度,也就是说,离群点往往会有更高的敏感度,
从而相比正常的点拥有更高的进入coreset的概率。这种现象在实际应用中可能会导致coreset的性能下降,我们希望coreset中包含更多的正常点。
现有的鲁棒 coreset 构建方法通常依赖于简单的均匀采样,并且仅在离群值的数量是输入规模的常数因子的情况下才有效。


\section{本文贡献}

在本文中,我们使用了一个统一的完全动态鲁棒 coreset 框架,适用于一类称为连续且有界(CnB)学习的优化问题。
这类学习问题涵盖了机器学习中的一类广泛优化目标。
简单来说,“CnB”要求优化目标是一个连续函数(例如,平滑或Lipschit连续),同时解决方案被限制在一个有界区域内。
我们强调这种“有界”假设在实际机器学习场景中是非常自然的。
为了阐明这一点,可以考虑一个迭代算法(例如常见的梯度下降或EM算法)来优化某个目标;除了最初的几轮,问题的解总是限制在一个局部区域内。
此外,在动态情况中将解决方案范围限定在有界区域也是合理的,因为单次更新(插入或删除)数据不太可能显著改变解。

本文沿用了\citet{Wang2021RobustAF}的coreset框架,首先,我们假设存在一个用于给定 CnB 优化目标的普通 coreset 构建方法 A(不考虑离群值)。
算法思路是将输入数据分为两部分:“疑似正常点”和“疑似离群点”,其中这两部分的大小比例是一个参数 λ。对于“疑似正常点”,
我们运行方法 A(作为一个黑箱);对于“疑似离群点”,我们直接随机均匀地抽取一个小样本;
最后,我们证明这两部分共同构成一个鲁棒的 coreset。
这个框架也可以在动态数据情况下高效地在“Merge and Reduce”框架中实现(尽管原始的合并和减少框架并未设计用于包含离群值的情况)[BS80, HM04]。
该框架的一个有趣特性是,如果动态数据中离群值的比例发生变化,我们可以轻松调整参数 λ 以动态更新我们的 coreset。

本文的另一贡献是,对于“疑似离群点”的随机采样,证明其采样复杂度与参数空间的doubling dimension有关,相比于之前的工作,我们的结果在参数空间有较好性质时,能得到更优的结果。
先前的结果中,对于“疑似离群点”的随机采样复杂度通常与数据空间的vc-dimension有关,这在实际应用中可能过大,甚至难以计算。




\section{符号定义}


我们介绍本文中使用的几个重要符号。假设 $\mathcal{P}$ 是参数空间。设 $X$ 为输入数据集,包含 $n$ 个项,位于度量空间 $X$ 中,每个 $x \in X$ 有一个权重 $w(x) \geq 0$。此外,我们用 $(X, z)$ 表示一个具有 $z$ 个离群值的给定实例 $X$。

我们始终使用 $| \cdot |$ 和 $[[ \cdot ]]$ 分别表示给定数据集的数据项数量和总权重。我们考虑的机器学习问题,其目标函数是 $X$ 上代价的加权和,即:
\begin{equation}
f(\theta, X) := \sum_{x \in X} w(x) f(\theta, x).
\label{eq:loss}
\end{equation}

其中,$f(\theta, x)$ 是参数向量 $\theta \in P$ 对应的由 $x$ 贡献的非负代价。目标是找到一个合适的 $\theta$ 使得目标函数 $f(\theta, X)$ 最小化。通常我们假设 $X$ 中的每个 $x$ 都有单位权重(即,$w(x) = 1$),并且很容易将我们的方法扩展到加权情况。

给定 $X$ 中预定数量的离群值 $z \in \mathbb{Z}^+$(对于加权情况,“$z$”指的是离群值的总权重),我们定义“鲁棒”目标函数:
\begin{equation}
f_z(\theta, X) := \min_{O \subset X, |O| = z} f(\theta, X \setminus O).
\label{eq:robust loss}
\end{equation}

实际上,上述定义~\eqref{eq:robust loss}来自于“修剪”思想,已被广泛应用于鲁棒优化问题。(引用)

接下来,我们给出CnB问题的形式定义。函数 $g: \mathcal{P} \rightarrow \mathbb{R}$ 是 $\alpha$-Lipschitz 连续的,
如果对于任意 $\theta_1, \theta_2 \in P$,有 $|g(\theta_1) - g(\theta_2)| \leq \alpha \|\Delta \theta\|$,
其中 $\Delta \theta = \theta_1 - \theta_2$ 且 $\|\cdot\|$ 是 $P$ 中某个指定的范数。
后文中为了方便,我们往往记为$|g(\theta_1) - g(\theta_2)| \leq \alpha \|\increment \theta\| = \xi(\increment \theta)$,其中 $\xi(\cdot)$ 是一个线性函数。


\begin{definition}[CnB 问题]
  令 $\alpha > 0$,$\ell > 0$,并且 $\tilde{\theta} \in \mathcal{P}$。用 $\mathbb{B}(\tilde{\theta}, \ell)$ 表示在
  参数空间 $\mathcal{P}$ 中以 $\tilde{\theta}$ 为中心、半径为 $\ell$ 的球。

如果目标函数~\eqref{eq:loss} 满足以下条件,则称其为具有参数 $(\alpha, \ell, \tilde{\theta})$ 的 CnB 学习问题:

(i)对于任何 $x \in X$,损失函数 $f(\cdot, x)$ 是 $\alpha$-Lipschitz 连续的。

(ii)$\theta$ 始终限制在 $\mathbb{B}(\tilde{\theta}, \ell)$ 内。
\label{def:CnB}
\end{definition}

我们定义CnB问题中的coreset如下:

\begin{definition}[$\varepsilon$-coreset]
  令 $\epsilon > 0$。给定一个数据集 $X \subseteq \mathcal{X}$ 
  和目标函数 $f(\theta, X)$,如果加权集合 $C \subseteq \mathcal{X}$ 满足对于
  任意 $\theta \in \mathbb{B}(\tilde{\theta}, \ell)$,都有
  \begin{equation*}
  |f(\theta, C) - f(\theta, X)| \leq \varepsilon f(\theta, X),
  \end{equation*}
则称 $C$ 是 $X$ 的一个 $\varepsilon$-coreset。
\label{def:coreset}
\end{definition}


如果 $C$ 是 $X$ 的 $\epsilon$-coreset,我们可以在 $C$ 上运行现有的优化算法,从而获得一个近似解。显然,我们期望 $C$ 的大小尽可能小。
根据定义\ref{def:coreset},我们还定义了相应的鲁棒 coreset。


\begin{definition}[鲁棒coreset]
  令 $\epsilon > 0$,$\beta \in [0, 1]$。给定数据集 $X \subseteq \mathcal{X}$ 
  和目标函数 $f(\theta, x)$,如果加权集合 $C \subseteq \mathcal{X}$ 满足对于任意 $\theta \in \mathbb{B}(\tilde{\theta}, \ell)$,都有
  \begin{equation*}
  (1 - \epsilon) f_{(1+\beta)z}(\theta, X) \leq f_z(\theta, C) \leq (1 + \epsilon) f_{(1-\beta)z}(\theta, X),
  \end{equation*}

则称 $C$ 是 $X$ 的一个 $(\beta, \epsilon)$-鲁棒 coreset。
\label{def:robust coreset}
\end{definition}


大致来说,如果我们在 $C$ 上获得一个近似解 $\theta' \in \mathcal{P}$,那么该解
在原始输入数据 $X$ 上的效果也能得到保持。参数 $\beta$ 表示在使用 $\theta'$ 作为 $X$ 的解时,离群值数量的误
差。如果我们设定 $\beta = 0$,这意味着我们允许离群值数量没有误差。

在第 B 节中,我们将详细讨论这种从 $C$ 到 $X$ 转换的损失(在损失函数值和离群值数量方面)。?

% 本文的其余部分组织如下。在第 3 节中,我们介绍我们的鲁棒 coreset 框架,并展示如何在完全动态环境中实现它。在第 4 节中,我们提出了两种不同的普通 coreset(没有离群值)构建方法,用于 CnB 学习问题,这些方法可以作为第 3 节中鲁棒 coreset 框架中的黑箱使用。最后,在第 5 节中,我们展示了我们的 coreset 方法在实践中的应用。

% \subsection{二级节标题}

% \subsubsection{三级节标题}

% \paragraph{四级节标题}

% \subparagraph{五级节标题}

% 本模板 \pkg{ustcthesis} 是中国科学技术大学本科生和研究生学位论文的 \LaTeX{}
% 模板, 按照《中国科学技术大学研究生学位论文撰写手册》(最近在修订中,以下简称《撰写手册》)和
% 《\href{https://www.teach.ustc.edu.cn/?attachment_id=13867}
% {中国科学技术大学本科毕业论文(设计)格式}》的要求编写。

% Lorem ipsum dolor sit amet, consectetur adipiscing elit, sed do eiusmod tempor
% incididunt ut labore et dolore magna aliqua.
% Ut enim ad minim veniam, quis nostrud exercitation ullamco laboris nisi ut
% aliquip ex ea commodo consequat.
% Duis aute irure dolor in reprehenderit in voluptate velit esse cillum dolore eu
% fugiat nulla pariatur.
% Excepteur sint occaecat cupidatat non proident, sunt in culpa qui officia
% deserunt mollit anim id est laborum.



% \section{脚注}

% Lorem ipsum dolor sit amet, consectetur adipiscing elit, sed do eiusmod tempor
% incididunt ut labore et dolore magna aliqua.
% \footnote{Ut enim ad minim veniam, quis nostrud exercitation ullamco laboris
%   nisi ut aliquip ex ea commodo consequat.
%   Duis aute irure dolor in reprehenderit in voluptate velit esse cillum dolore
%   eu fugiat nulla pariatur.}
