% !TeX root = ../main.tex

\chapter{绪论}

在大数据时代,随着数据量的迅速增加,我们往往需要更低的时间复杂度算法来完成任务。此外,
在一些实际场景中,我们的数据是动态的,即数据集合会随着时间的推移而发生变化。显然,如果数据量极其庞大,
当训练数据动态变化时,重新训练整个模型的代价是不可行的。

为了解决这些问题,一个很自然的想法是构建一个小规模的训练集,以便我们可以在小规模集合而不是所有数据上运行现有的算法。
Coreset,最初出现于计算几何研究中的概念,已成为许多大规模机器学习问题中广泛使用的数据压缩技术。作为一种简洁的
数据压缩技术,Coreset还具有一些优秀的性质。例如,Coreset通常是可组合的,因此可以应用于分布式计算环境之中。
此外,它往往能在流算法和包含数据增删的动态算法中取得较好的表现。

然而,现有的Coreset算法仍然有较大的改进空间。一个主要的瓶颈是,多数Coreset算法容易受到异常点(离群点)的影响
而导致性能下降,而现实数据集中往往有许多噪声。许多现有研究也表明,只需在训练样本中加入少数攻击点,就可能导致原模型的效果显著下降。
因此,如何设计一个鲁棒性更好的Coreset算法是一个重要的研究方向。

为了简要说明现有的coreset方法对离群值的敏感性,我们可以以一种基于采样的coreset算法为例。这个算法需要对每个数据点计算一个“敏感度”,
该敏感度衡量了该数据项对整个数据集的重要程度;然而,它往往会给远离数据多数部分的点分配高敏感度,也就是说,离群点往往会有更高的敏感度,
从而相比正常的点拥有更高的进入Coreset的概率。这种现象在实际应用中可能会导致Coreset的性能下降,我们希望coreset中包含更多的正常点。
现有的鲁棒 coreset 构建方法通常依赖于简单的均匀采样,并且仅在离群值的数量是输入规模的常数因子的情况下才有效。


\section{本文贡献}


% \subsection{二级节标题}

% \subsubsection{三级节标题}

% \paragraph{四级节标题}

% \subparagraph{五级节标题}

本模板 \pkg{ustcthesis} 是中国科学技术大学本科生和研究生学位论文的 \LaTeX{}
模板, 按照《中国科学技术大学研究生学位论文撰写手册》(最近在修订中,以下简称《撰写手册》)和
《\href{https://www.teach.ustc.edu.cn/?attachment_id=13867}
{中国科学技术大学本科毕业论文(设计)格式}》的要求编写。

Lorem ipsum dolor sit amet, consectetur adipiscing elit, sed do eiusmod tempor
incididunt ut labore et dolore magna aliqua.
Ut enim ad minim veniam, quis nostrud exercitation ullamco laboris nisi ut
aliquip ex ea commodo consequat.
Duis aute irure dolor in reprehenderit in voluptate velit esse cillum dolore eu
fugiat nulla pariatur.
Excepteur sint occaecat cupidatat non proident, sunt in culpa qui officia
deserunt mollit anim id est laborum.



\section{脚注}

Lorem ipsum dolor sit amet, consectetur adipiscing elit, sed do eiusmod tempor
incididunt ut labore et dolore magna aliqua.
\footnote{Ut enim ad minim veniam, quis nostrud exercitation ullamco laboris
  nisi ut aliquip ex ea commodo consequat.
  Duis aute irure dolor in reprehenderit in voluptate velit esse cillum dolore
  eu fugiat nulla pariatur.}
