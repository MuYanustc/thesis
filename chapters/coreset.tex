\chapter{Coreset框架}

我们构建鲁棒coreset的想法来自以下直觉:
在理想情况下,如果我们知道哪些是正常点,哪些是离群点,我们可以简单地分别为它们构建coreset。
实际上,虽然我们无法得到如此明确的分类,但CnB条件(定义\ref{def:CnB})可以指导我们获得“粗略”的分类。
我们可以证明这种混合框架可以生成一个鲁棒coreset。

\section{框架细节}

首先利用$\tilde{\theta}$对数据进行“粗略”的分类。我们根据 $f(\tilde{\theta}, x)$ 的值将数据分为两部分:$X_{\text{si}}$和$X_{\text{so}}$,前者是“可能”的正常点
,后者是“可能”的离群点。令 $\varepsilon_0 := \min \left\{ \frac{\epsilon}{16}, \frac{\epsilon \cdot \inf_{\theta \in \mathcal{B}(\tilde{\theta}, \epsilon)} f_z(\theta, X)}{16(n - z) \alpha \ell} \right\}$,并且令 $\tilde{z} := (1 + 1/\epsilon_0) z$;$x_z \in X$ 是在 $X$ 中具有第 $\tilde{z}$ 大成本 $f(\tilde{\theta}, x)$ 的点。

我们令 $\tau = f(\tilde{\theta}, x_z)$,于是我们得到集合
\begin{equation}
\{ x \in X \mid f(\tilde{\theta}, x) \geq \tau \}.
\end{equation}

该集合即为 $X_{\text{so}}$,将剩余的点作为“可能”的正常点 $X_{\text{si}} := X \setminus X_{\text{so}}$。下面我们利用这两个集合构建coreset。

假设我们有一个黑盒的coreset方法$\mathcal{A}$。我们通过方法 $\mathcal{A}$ 为 $X_{si}$ 构建一个 $\epsilon_1$-coreset ($\epsilon_1 = \epsilon / 4$),并通过设置 $\delta = \frac{\beta \epsilon_0}{1 + \epsilon_0}$ 为 $X_{so}$ 取一个 $\delta$-样本。

我们将这两个集合分别记为 $C_{si}$ 和 $C_{so}$。如果我们设置 $\beta = 0$(即,不要求离群值数量上的误差),我们直接将 $X_{so}$ 的所有点作为 $C_{so}$。最后,我们返回 $C = C_{si} \cup C_{so}$ 作为鲁棒 coreset。

\begin{theorem}
    对于一个CnB问题,我们可以通过构建上述框架,其以概率$1-\eta$返回一个大小为:
    \begin{equation*}
        |C_{si}|+\min\left\{O\left(\frac{1}{\beta^2\varepsilon^2}(\text{vcdim}+\log \frac 1 \eta)\right),O\left(\frac{1}{\beta^2\varepsilon_0^2}(\text{ddim}\log\frac{1}{\varepsilon_0}+\log \frac{1}{\eta})\right),O\left(\frac z {\varepsilon_0}\right)\right\}
    \end{equation*}
的$(\beta,\epsilon)$-coreset。特殊的,当$\beta = 0$,大小为$|C_{si}|+O\left(\frac{z}{\varepsilon_0}\right)$
\end{theorem}
%cite wzx
本文在之前工作的基础上,基于\ref{thm:uniform final}优化了$C_{si}$的大小,使其理论下限不再依赖于vcdim。