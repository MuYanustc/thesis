\chapter{均匀采样}

均匀采样作为一种简单高效的采样方法,在各种各样的coreset算法中都有广泛应用。
在本章中,我们将介绍均匀采样的一些性质,为本文后续的证明铺垫。

在此之前,先介绍一些关于范围空间的基本概念。

\begin{definition}[f-诱导的范围空间]
    假设 $\mathcal{X}$ 是一个任意度量空间。给定 $X$ 上的损失函数 $f(\theta, X)$ 如公式~\eqref{eq:loss} 所示,我们定义
    \begin{equation}
    \mathcal{R} = \left\{ \{ x \in \mathcal{X} : f(\theta, x) \leq r \} \mid \forall r \geq 0, \forall \theta \in \mathcal{P} \right\},
    \end{equation}

则 $(\mathcal{X}, \mathcal{R})$ 称为 $f$-诱导的范围空间。
每个 $R \in \mathcal{R}$ 称为 $\mathcal{X}$ 的一个范围。

\end{definition}

以下的“$\delta$-样本”概念来自 VC 维度理论。

给定范围空间 $(\mathcal{X}, \mathcal{R})$,设 $C$ 和 $X$ 是 $\mathcal{X}$ 的两个有限子集。假设 $\delta \in (0, 1)$。如果 $C \subseteq X$ 并且对于任意 $R \in \mathcal{R}$ 都有
\begin{equation}
\left| \frac{|X \cap R|}{|X|} - \frac{|C \cap R|}{|C|} \right| \leq \delta,
\end{equation}

则称 $C$ 是 $X$ 的一个 $\delta$-样本。

记vcdim为范围空间 $(\mathcal{X}, \mathcal{R})$ 的 VC-dimension。
那么我们可以通过从 $X$ 中均匀采样 $O\left(\frac{1}{\delta^2} \left(\text{vcdim} + \log \frac{1}{\eta}\right)\right)$ 个点
,以概率 $1 - \eta$ 获得一个 $\delta$-样本 \citep{LLS01}。$\text{vcdim}$ 的值取决于函数“$f$”。
例如,如果“$f$”是 $\mathbb{R}^d$ 中逻辑回归的损失函数,
那么 $\text{vcdim}$ 可以达到 $\Theta(d)$ \citep{MSSW18}。
以下定理表明,如果 $z$ 是 $n$ 的常数因子,那么 $\delta$-样本可以作
为鲁棒 coreset。请注意,在以下定理中,目标函数 $f$ 可以是任何函数,而不仅仅是定义 1 中所述的函数。

% 引用 wzx
\begin{theorem}
    设 $(X, z)$ 是鲁棒学习问题 ~\eqref{eq:robust loss} 的一个实例。
    如果 $C$ 是 $f$-诱导范围空间中的 $X$ 的一个 $\delta$-样本,
    我们为每个 $c \in C$ 赋值权重 $w(c) = \frac{n}{|C|}$。则我们有
    \begin{equation*}
    f_{z+\delta n}(\theta, X) \leq f_z(\theta, C) \leq f_{z-\delta n}(\theta, X)
    \end{equation*}
    对于任意 $\theta \in \mathcal{P}$ 和任
    意 $\delta \in (0, z/n)$。
    特别地,如果 $\delta = \beta z / n$,
    则 $C$ 是 $X$ 的一个 $(\beta, 0)$-鲁棒 coreset,
    并且 $C$ 的大小为 $O\left(\frac{1}{\beta^2}\left(\frac{n}{z}\right)^2\left(\text{vcdim} + \log \frac{1}{\eta}\right)\right)$。
\end{theorem}

\begin{corollary}
    对于某个固定参数$\theta \in \mathcal{P}$, 如果我们从 $X$ 中均
    匀采样 $O\left(\frac{1}{\delta^2}  \log \frac{2}{\eta}\right)$ 个点,
    那么我们以概率 $1 - \eta$ 获得一个子诱导空间 $(\mathcal{X}, \mathcal{R}')$ 中的 $\delta$-样本。
    \label{cor:fixed theta}
\end{corollary}   

这个推论可以通过考虑固定$\theta$的特殊情况导出。如果此时的$f$满足\ref{def:CnB}中的CnB条件,那么我们可以利用Lipcshitz连续性
得到一些更强的结果。

\begin{theorem}
    如果 $f$ 满足定义~\ref{def:CnB} 中的CnB条件,那么我们可以从 $X$ 中均匀采样 $O\left(\frac{1}{\delta^2} \left(\text{ddim}\log \frac 1 \varepsilon + \log \frac{1}{\eta}\right)\right)$ 个点,
    以概率 $1 - \eta$ 获得一个($\beta, \epsilon$)-coreset。其中$\text{ddim}$是参数空间$\mathcal{P}$的加倍维度,$\beta = \frac{\delta n}{z}$。
\end{theorem}
\begin{proof}
    在证明该命题前,有一些需要用到的定义和命题需要给出。

    \begin{definition}[$\varepsilon$-net]
        对于一个集合$X$,如果存在一个子集$N \subseteq X$,使得对于任意$x \in X$,都存在一个$y \in N$,使得$d(x, y) \leq \varepsilon$,那么称$N$是$X$的一个$\varepsilon$-net。
    \end{definition}

    \begin{proposition}
        对于一个空间$\mathcal{X}$,如果$\mathcal{X}$的加倍维度ddim为$d$,那么$\mathcal{X}$
        中的$\mathbb{B}(\tilde{\theta}, \ell)$对应的
        $\varepsilon l$-net的大小为$O\left(\frac{1}{\varepsilon^d}\right)$。
        \label{prop:net}
    \end{proposition}

    \begin{proposition}[Union bound]
        一般地,对于可数个事件 $A_1, A_2, A_3, \dots ,A_n$,我们有
        \begin{equation*}
            \mathbb{P} \left( \bigcup_{i=1}^{n} A_i \right) \leq \sum_{i=1}^{n} \mathbb{P}(A_i).
        \end{equation*}
        \label{prop:union bound}
    \end{proposition}
    加倍维度刻画了空间的一些性质,最直观的反应在$\varepsilon$-net的大小上。我们可以在
$\mathbb{B}(\tilde{\theta}, \ell)$中随机采样$O\left(\frac{1}{\delta^2}\left(\log \frac{2|\mathbb{B}^{\varepsilon \ell}|}{\eta}\right)\right)$个点,记为$\mathcal{C}$
其中$\mathbb{B}^{\varepsilon \ell}$是$\mathbb{B}(\tilde{\theta}, \ell)$的一个$\epsilon l$-net。

根据命题\ref{prop:union bound}和推论\ref{cor:fixed theta}我们可以得到:
\begin{equation}
    \forall \theta \in \mathbb{B}^{\varepsilon \ell}, P\left[\mathcal{C}\text{是对应}(X,\mathcal{R}’)\text{的}\delta-\text{样本}\right]>1-\eta.
\end{equation}

这个结论意味着对于$\theta' \in \mathbb{B}^{\varepsilon\ell}$,带入$\beta = \frac{\delta n }{z}$,我们有
\begin{equation*}
    (1-\varepsilon)f_{(1+\beta )z}(\theta, X) <
    f_{(1+\beta )z}(\theta, X) \leq 
    f_z(\theta, \mathcal{C}) \leq f_{(1-\beta)z}(\theta, X) <
    (1+\varepsilon)f_{(1-\beta)z}(\theta, X).
\end{equation*}

而对于$\theta\notin\mathbb{B}^{\varepsilon \ell}$的情况,我们取$\varepsilon_0=min\left\{\frac{\varepsilon}{2}, \frac{\varepsilon}{2n\xi(\ell)}\inf_{\theta\in\mathbb{B}(\tilde{\theta},\ell)}f_z(\theta,X)\right\}$
记$\theta_t$为$\varepsilon l$-net中距$\theta'$最近的点,我们有:
\begin{align*}
    f_z(\theta',\mathcal{C})&\leq f_z(\theta_t,\mathcal{C})+n\xi(\varepsilon_0\ell)\\
    &\leq f_{z-\delta n}(\theta_t,X)+n\varepsilon_0\xi(\ell)\\
    &\leq f_{z-\delta n}(\theta',X)+2n\varepsilon_0\xi(\ell).
\end{align*}
当$\frac\varepsilon 2 \le \frac{\varepsilon}{2n\xi(\ell)}\inf_{\theta\in\mathbb{B}(\tilde{\theta},\ell)}f_z(\theta,X)$,即$n\xi(\ell)\le\inf_{\theta\in\mathbb{B}(\tilde{\theta},\ell)f_z(\theta,X)}$时,有:
\begin{align*}
    f_z(\theta',\mathcal{C})&\leq f_{z-\delta n}(\theta',X)+\varepsilon f_{z-\delta n}(\theta,X)\\
    &\leq f_{z-\delta n}(\theta',X)+\varepsilon \inf_{\theta\in\mathbb{B}(\tilde{\theta},\ell)}f_z(\theta,X)\\
    &\leq f_{z-\delta n}(\theta',X)+\varepsilon f_{z-\delta n}(\theta',X)=(1+\varepsilon)f_{z-\delta n}(\theta',X).
\end{align*}
当$\frac\varepsilon 2 \ge \frac{\varepsilon}{2n\xi(\ell)}\inf_{\theta\in\mathbb{B}(\tilde{\theta},\ell)}f_z(\theta,X)$,即$n\xi(\ell)\ge\inf_{\theta\in\mathbb{B}(\tilde{\theta},\ell)f_z(\theta,X)}$时,有:
\end{proof}
