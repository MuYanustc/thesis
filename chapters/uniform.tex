\chapter{均匀采样}

均匀采样作为一种简单高效的采样方法,在各种各样的coreset算法中都有广泛应用。
在本章中,我们将介绍均匀采样的一些性质,为本文后续的证明铺垫。

在此之前,先介绍一些关于范围空间的基本概念。

\begin{definition}[f-诱导的范围空间]
    假设 $\mathcal{X}$ 是一个任意度量空间。给定 $X$ 上的损失函数 $f(\theta, X)$ 如公式~\eqref{eq:loss} 所示,我们定义
    \begin{equation}
    \mathcal{R} = \left\{ \{ x \in \mathcal{X} : f(\theta, x) \leq r \} \mid \forall r \geq 0, \forall \theta \in \mathcal{P} \right\},
    \end{equation}

则 $(\mathcal{X}, \mathcal{R})$ 称为 $f$-诱导的范围空间。
每个 $R \in \mathcal{R}$ 称为 $\mathcal{X}$ 的一个范围。

\end{definition}

以下的“$\delta$-样本”概念来自 VC 维度理论。

给定范围空间 $(\mathcal{X}, \mathcal{R})$,设 $C$ 和 $X$ 是 $\mathcal{X}$ 的两个有限子集。假设 $\delta \in (0, 1)$。如果 $C \subseteq X$ 并且对于任意 $R \in \mathcal{R}$ 都有
\begin{equation}
\left| \frac{|X \cap R|}{|X|} - \frac{|C \cap R|}{|C|} \right| \leq \delta,
\end{equation}

则称 $C$ 是 $X$ 的一个 $\delta$-样本。

记vcdim为范围空间 $(\mathcal{X}, \mathcal{R})$ 的 VC-dimension。
那么我们可以通过从 $X$ 中均匀采样 $O\left(\frac{1}{\delta^2} \left(\text{vcdim} + \log \frac{1}{\eta}\right)\right)$ 个点
,以概率 $1 - \eta$ 获得一个 $\delta$-样本 \citep{LLS01}。$\text{vcdim}$ 的值取决于函数“$f$”。
例如,如果“$f$”是 $\mathbb{R}^d$ 中逻辑回归的损失函数,
那么 $\text{vcdim}$ 可以达到 $\Theta(d)$ \citep{MSSW18}。
以下定理表明,如果 $z$ 是 $n$ 的常数因子,那么 $\delta$-样本可以作
为鲁棒 coreset。请注意,在以下定理中,目标函数 $f$ 可以是任何函数,而不仅仅是定义 1 中所述的函数。