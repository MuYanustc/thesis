% !TeX root = ../main.tex

\ustcsetup{
  keywords  = {核心集, 加倍维度, 加倍空间,连续有界学习问题,k聚类问题},
  keywords* = {coreset, doubling dimension, doubling space, continuous bounded learning problems, k-means clustering problems},
}

\begin{abstract}
  在许多机器学习任务中,处理大规模数据的常见方法是构建一个小的数据集,
  例如核心集,这可以有效地表示原始输入的性质。
  然而,真实世界的数据集通常包含异常值,而大多数现有的核心集构建方法对异常值不具备韧性
  (特别是异常值可能由对抗攻击出现在问题空间中)。
  在本文中,我们使用了一种可用于连续且有界学习问题(包含异常值)的鲁棒核心集方法,
  这类问题包括了机器学习中广泛应用的优化目标,例如逻辑回归和k-means聚类。
  此外,本文的核心集框架可以在完全动态的环境中高效维护。
  另一个亮点是,本文基于参数空间的加倍维度,针对核心集的大小进行优化,
  使其可以依赖于参数空间的倍增维数,而不是损失函数的VC维数,后者可能非常大甚至难以计算。
  最后,我们在真实世界的数据集上进行了实验,以评估我们提出的鲁棒核心集方法的有效性。
\end{abstract}

\begin{abstract*}
  In many machine learning tasks, a common approach for handling large-scale data is to construct a small dataset, 
  such as a coreset, which can effectively capture the properties of the original input. However, 
  real-world datasets often contain outliers, and most existing coreset construction methods 
  are not resilient to outliers (particularly, outliers may be arbitrarily located in the space by adversarial 
  attackers). In this paper, we propose a robust coreset method that can be applied to continuous and 
  bounded learning problems (including outliers), which encompass a broad range of widely-used optimization 
  objectives in machine learning, such as logistic regression and k-means clustering. Moreover, 
  the coreset framework presented in this paper can be efficiently maintained in a fully dynamic environment. 
  Another highlight is that the size of the coreset can be optimized based on the doubling dimension of the 
  parameter space, rather than the VC dimension of the loss function, which can be very large or even challenging 
  to compute. Finally, we conduct experiments on real-world datasets to evaluate the effectiveness of the 
  proposed robust coreset method.
\end{abstract*}
