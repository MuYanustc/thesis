\chapter{核心集框架}

我们构建鲁棒核心集的想法符合以下直觉:
在理想情况下,如果我们知道哪些是正常点,哪些是离群点,我们可以简单地分别为它们构建核心集。
实际上,虽然我们无法得到如此明确的分类,但CnB条件(定义\ref{def:CnB})可以指导我们获得“粗略”的分类。
可以证明\cite{Wang2021RobustAF},这种混合框架可以生成一个鲁棒核心集。

\section{框架细节}

首先利用$\tilde{\theta}$对数据进行“粗略”的分类。我们根据 $f(\tilde{\theta}, x)$ 的值将数据分为两部分:$X_{\text{si}}$和$X_{\text{so}}$,前者是“可能”的正常点
,后者是“可能”的离群点。令 $\varepsilon_0 := \min \left\{ \frac{\varepsilon}{16}, \frac{\varepsilon \cdot \inf_{\theta \in \mathcal{B}(\tilde{\theta}, \varepsilon)} f_z(\theta, X)}{16(n - z) \alpha \ell} \right\}$,并且令 $\tilde{z} := (1 + 1/\varepsilon_0) z$;$x_z \in X$ 是在 $X$ 中具有第 $\tilde{z}$ 大成本 $f(\tilde{\theta}, x)$ 的点。

我们令 $\tau = f(\tilde{\theta}, x_{\tilde{z}})$,于是我们得到集合
\begin{equation}
\{ x \in X \mid f(\tilde{\theta}, x) \geq \tau \}.
\end{equation}

该集合即为 $X_{\text{so}}$,将剩余的点作为“可能”的正常点 $X_{\text{si}} := X \setminus X_{\text{so}}$。下面我们利用这两个集合构建核心集。

\textbf{算法:}假设我们有一个黑盒的核心集方法$\mathcal{A}$。我们通过方法 $\mathcal{A}$ 为 $X_{si}$ 构建一个 $\varepsilon_1$-coreset ($\varepsilon_1 = \varepsilon / 4$),并通过设置 $\delta = \frac{\beta \varepsilon_0}{1 + \varepsilon_0}$ 为 $X_{so}$ 取一个 $\delta$-样本。

我们将这两个集合分别记为 $C_{si}$ 和 $C_{so}$。如果我们设置 $\beta = 0$(即,不要求离群值数量上的误差),我们直接将 $X_{so}$ 的所有点作为 $C_{so}$。最后,我们返回 $C = C_{si} \cup C_{so}$ 作为鲁棒 核心集。

有了第二章关于均匀采样的讨论,结合\cite{Wang2021RobustAF}的理论证明,我们可以自然的得到以下定理:
\begin{theorem}
    对于一个CnB问题,我们可以通过构建上述框架,其以概率$1-\eta$返回一个大小为:
    \begin{equation*}
        |C_{si}|+\min\left\{O\left(\frac{1}{\beta^2\varepsilon^2}(\text{vcdim}+\log \frac 1 \eta)\right),O\left(\frac{1}{\beta^2\varepsilon_0^2}(\text{ddim}\log\frac{1}{\varepsilon_0}+\log \frac{1}{\eta})\right),O\left(\frac z {\varepsilon_0}\right)\right\}
    \end{equation*}
的$(\beta,\varepsilon)$-coreset。特殊的,当$\beta = 0$,大小为$|C_{si}|+O\left(\frac{z}{\varepsilon_0}\right)$
\end{theorem}
%cite wzx
本文在之前工作的基础上,基于定理\ref{thm:uniform final}优化了$C_{si}$的大小,使其理论下限不严格依赖于vcdim。

\section{动态情况}

本文使用的核心集框架的另一个特点是其适应于动态数据流。在动态情况下,数据集$X$会不断变化,我们需要不断更新核心集。

标准的 $\varepsilon$-coreset 通常具有两个重要性质。如果 $C_1$ 和 $C_2$ 分别是两个不相交集合 $X_1$ 
和 $X_2$ 的 $\varepsilon$-coreset,那么它们的并集 $C_1 \cup C_2$ 应该是 $X_1 \cup X_2$ 的 $\varepsilon$-coreset。
同时,如果 $C_1$ 是 $C_2$ 的 $\varepsilon_1$-coreset 且 $C_2$ 是 $C_3$ 的 $\varepsilon_2$-coreset,
那么 $C_1$ 应该是 $C_3$ 的 $(\varepsilon_1 + \varepsilon_2 + \varepsilon_1 \varepsilon_2)$-coreset。
基于这两个性质,可以使用“merge-and-reduce”技术 \citep{DBLP:conf/stoc/Har-PeledM04,DBLP:conf/esa/HenzingerK20} 为增量数据流构建一个核心集。


大致来说,“merge-and-reduce”技术使用一系列“桶”来维护输入流数据的 核心集,并以自底向上的方式合并这些桶。
然而,在有离群值的情况下直接使用这种策略需要一些额外设计,因为我们无法确定每个桶中的离群值数量。
混合鲁棒核心集框架的一个优点是,我们可以通过使用一个大小为 $O(n)$ 的辅助表 $\mathcal{L}$ 
以及“merge-and-reduce”技术轻松解决这一障碍(即使在没有离群值的情况下,维护一个完全动态的 coreset 已经需要 $\Omega(n)$ 的空间 \cite{DBLP:conf/esa/HenzingerK20})。
我们在下面简要介绍具体算法。

回顾一下,我们将输入数据 $X$ 分为两部分:$n - \tilde{z}$ 个“可能的正常
点”和 $\tilde{z}$ 个“可能的离群点”,其中 $\tilde{z} = (1 + 1/\varepsilon_0)z$。

对于第一部分,我们只需直接应用"merge-and-reduce"技术来获得一个完全动态核心集 $C_{si}$;对于第二部分,我们可以只取一个 $\delta$-样本或取整个集合(如果我们要求 $\beta = 0$),并将其表示为 $C_{so}$。

此外,我们在一个表 $\mathcal{L}$ 中记录关键值 $x.value = f(\tilde{\theta}, x)$ 及其
在merge-and-reduce树中的位置 $x.position$,对于 $X$ 中的每个 $x$;它们按 $x.value$ 排序。
为了处理动态更新(例如删除和插入),我们还维护一个指向数据项 $x_z$ 的关键指针 $p$(
回忆 $x_z$ 是 $X$ 中具有第 $\tilde{z}$ 大成本 $f(\tilde{\theta}, x)$ 的点)。

当一个新的数据项 $x$ 进入或现有数据项 $x$ 将被删除时,我们只需要将其与 $f(\tilde{\theta}, x_z)$ 比较,以决定是更新 $C_{si}$ 还是 $C_{so}$;更新后,我们还需要更新 $x_z$ 和表 $\mathcal{L}$ 中的指针 $p$。如果离群值数量 $z$ 发生变化,我们只需要首先更新 $x_z$ 和 $p$,然后更新 $C_{si}$ 和 $C_{so}$(例如,如果 $z$ 增加,我们只需要从 $C_{so}$ 中删除一些项,并将一些项插入 $C_{si}$)。为了实现这些更新操作,我们还选择了一个桶作为“热桶”,它充当执行所有数据移动的传递工具。

为了整体上获得一个 $\varepsilon$-coreset,我们需要在每个结点向上更新时构建一个大小为 $M(\varepsilon / \log n)$ 的 核心集 \cite{DBLP:journals/jacm/AgarwalHV04}。我们用 $M$ 表示 $M(\varepsilon / \log n)$ 并假设我们可以在时间 $t(|X|)$ 内计算 $X$ 的 核心集 \cite{DBLP:books/daglib/0035668},我们针对增删改有以下结果。

\begin{enumerate}
    \item Insert(y)(插入数据点y)根据$y.value=f(\tilde{\theta},y)$插入表
    $\mathcal{L}$中。如果$y.value\geq p\rightarrow value$,则将$y$插入$C_{so}$,否则插入$C_{si}$。
    如果$y.value>f(\tilde{\theta},x_z)=p\rightarrow value$,则将p指向的元素加入热桶,更新$p\leftarrow p+1$;
    如果$y.value\leq p\rightarrow value$,将该元素加入热桶。
    最后从热桶自底向上更新整颗树。更新时间为$O\left(t(M)\log n\right)$。
    \item Delete(y)(删除数据点y)如果$y$是可能的正常点,那么将其从对应的桶$y.position$中删除,
    从这个桶更新树,最后从表$\mathcal{L}$中删除$y$。如果$y$是可能的离群点,直接从表$\mathcal{L}$中删除$y$,然后更新$p\leftarrow p-1$,
    最后删除$p$指向的点,从对应桶更新树。总时间开销是$O\left(t(M)\log n\right)$。
    \item Update(y)(更新数据点y)只需要运行Delete(y)和Insert(y)即可,运行时间为$O\left(t(M)\log n\right)$。
    \item Change($\increment$z)(改变离群点的数量)根据$\increment z$的正负分别运行$\tilde{z}(z+\increment z)-\tilde{z}(z)$次针对树的Insert或Delete操作。对应的更新时间为$O\left(\frac{\increment z}{\varepsilon}t(M)\log n\right)$。
\end{enumerate}

注意到上述算法中Insert(y)和Delete(y)在实际操作中的时间开销还需要加上查找和插入的时间,但这些时间开销都是$O(\log n)$的,因此不会影响总的时间复杂度。图\ref{fig:dyn}展示了动态情况下的核心集框架示意图。
\begin{figure}
    \centering
    \subcaptionbox[Short Subcaption]{%
        Merge-and-Reduce(无离群点时)%
        \label{subfig:sublabel1}%
    }
    [%
        0.45\textwidth % width of caption
    ]%
    {%
        \includegraphics[width=0.45\textwidth]%
        {./fig/merge-and-reduce.pdf}%
    }%
    \hfill
    \subcaptionbox[Short Subcaption]{%
        Merge-and-Reduce(有离群点时)%
        \label{subfig:sublabel2}%
    }
    [%
        0.45\textwidth % width of caption
    ]%
    {%
        \includegraphics[width=0.45\textwidth]%
        {./fig/merge-and-reduce-outliers.pdf}%
    }%
    \caption[Short Caption]{动态情况下的核心集框架}
    \label{fig:dyn}
\end{figure}