\chapter{黑盒方法}

正如前言,我们的coreset框架需要一个黑盒的coreset方法$\mathcal{A}$。
本章中我们提供两种不同的黑盒方法,并探讨其在CnB问题下的复杂度。

\section{基于重要性采样的coreset}

我们遵循基于重要性采样的方法 \citep{LS10}。
假设 $X = \{x_1, \cdots, x_n\}$。对于每个数据点 $x_i$,
它的敏感度 $\sigma_i = \sup_\theta \frac{f(\theta, x_i)}{f(\theta, X)}$ 
测量了其对整个输入数据 $X$ 的重要性。计算敏感度通常很困难,但敏感度的上界实际上已经足以构建 
coreset。假设 $s_i$ 是 $\sigma_i$ 的上界,并且令 $S = \sum_{i=1}^n s_i$。coreset 的构建如下。
我们从 $X$ 中采样一个子集 $C$,其中 $C$ 的每个元素是独立同分布采样的,
概率 $p_i = s_i / S$。我们为 $C$ 中的每个采样数据项 $x_i$ 赋值权重 $w_i = \frac{S}{s_i |C|}$。
最后,我们返回 $C$ 作为 coreset。


\begin{theorem}[\citep{BFL16}]。

令 vcdim 为由 $f(\theta, x)$ 诱导的范围空间的 VC 维度(或 shattering 维度)。如果 $C$ 的大小为
\begin{equation}
\Theta \left( \frac{S}{\varepsilon^2} \left( \text{vcdim} \cdot \log S + \log \frac{1}{\eta} \right) \right)
\end{equation}
则 $C$ 是一个 $\epsilon$-coreset,概率至少为 $1 - \eta$。
\label{thm:importance sampling}
\end{theorem}

因此,唯一剩下的问题是如何计算上界 $s_i$。回想一下,
我们假设我们的成本函数在定义 1 中是 $\alpha$-Lipschitz
(或 $\alpha$-平滑,$\alpha$-Lipschitz 连续 Hessian)。
也就是说,我们可以界定 $f(\theta, x_i)$ 和 $f(\tilde{\theta}, x_i)$ 之间的差异,
这样的界限可以帮助我们计算 $s_i$。
在第 D 节中,我们展示了计算 $s_i$ 等同于求解一个二次分式规划。
这种规划可以简化为一个半定规划(SDP)\citep{BT09},
可以在多项式时间内解决到任意所需的精度 \citep{GM12}。
我们用 $\text{T}(d)$ 表示 SDP 的求解时间,其中 $d$ 是数据点的维度。
因此,coreset 构建的总运行时间为 $O(n \cdot \text{T}(d))$。

%  这里补到前面,加倍维度是描述数据增长率的常用度量,也可以视为欧几里得维度的一般化。例如,d 维欧几里得空间的加倍维度为 $\Theta(d)$ \citep{CGMZ16}。
定理\ref{thm:importance sampling}的一个缺点是 coreset 的大小取决于由 $f(\theta, x)$ 诱导的 \text{vcdim}。
对于一些目标函数,vcdim 的值可能非常大或难以获得。
在这里,我们证明了对于一个连续且有界的成本函数,coreset 的大小可以与 \text{vcdim}无关;
相反,它取决于参数空间 $\mathcal{P}$ 的加倍维度 \text{ddim}。
定理\ref{thm:improved importance sampling}相对于定理\ref{thm:importance sampling}的主要优势在于我们不需要知道
由损失函数诱导的 VC 维度。另一方面,加倍维度通常更容易计算
(或估计),例如,给定实例在 $\mathbb{R}^d$ 中的加倍维度仅为 $\Theta(d)$,
即使问题的损失函数可能非常复杂。

定理\ref{thm:improved importance sampling}的另一个动机来自稀疏优化。设参数空间为 $\mathbb{R}^D$,
我们限制 $\theta$ 为 $k$-稀疏(即,至多有 $k$ 个非零项,且 $k \ll D$)。
很容易看出,$\theta$ 的定义域是 $\binom{D}{k}$ 个 $k$ 维子空间的并集,
因此其加倍维度为 $O(k \log D)\ll O(D)$。

\begin{theorem}
    对于一个CnB问题,令ddim为参数空间$\mathcal{P}$的加倍维度。
    那么运行上述基于重要性采样的coreset方法,我们以概率为$1-\eta$得到一个coreset,
    其大小为$\Theta\left(\frac{S}{\varepsilon^2}\left(\log S +\log \frac 1 \eta + \text{ddim}\log\frac{1}{\varepsilon}\right)\right)$。
    \label{thm:improved importance sampling}
\end{theorem}

\begin{corollary}
    对于一个固定的$\theta_0\in \mathbb{B}(\tilde{\theta},l)$,运行重要性采样的coreset方法,
    我们以概率$1-\eta$得到一个coreset,其大小为$\Theta\left(\frac{S}{\varepsilon^2}\left(\log S +\log \frac 1 \eta \right)\right)$。
    \label{cor:importance sampling with fixed theta}
\end{corollary}
\begin{proof}
    我们的证明思路类似于定理\ref{thm:uniform}的证明。同样的,考虑$\mathbb{B}(\tilde{\theta},l)$的$\varepsilon l$-net,并在
    $X\subset \mathcal{X}$上采样$\Theta\left(\frac{S}{\varepsilon^2}\left(\log S +\log \frac {\left|\mathbb{B}^{\varepsilon \ell}\right|} {\eta} \right)\right)$

    同样的,对于$\theta \in \mathbb{B}^{\varepsilon \ell}$,利用union bound和推论\ref{cor:importance sampling with fixed theta},以概率$1-\eta$是一个$\varepsilon$-coreset。

    对于$\theta' \in \mathbb{B}(\tilde{\theta},l)\setminus \mathbb{B}^{\varepsilon \ell}$,令$\theta_t$为$\mathbb{B}^{\varepsilon \ell}$中距离$\theta'$最近的点。利用三角不等式:
    \begin{align*}
        |f(\theta'&,X)-f(\theta',C)|\leq \\
        &|f(\theta',X)-f(\theta_t,X)|+|f(\theta_t,X)-f(\theta_t,C)|+|f(\theta',C)-f(\theta_t,C)|
    \end{align*}
    注意此处的$f(\theta,\cdot)$满足\ref{eq:loss}的带权形式。
    第一项不超过 $\varepsilon n \xi(\ell)$,因为 $\xi(\varepsilon \ell) \leq \epsilon \xi(\ell)$。
    中间项可以由 $\varepsilon (f(\theta, X) + n \xi(\ell))$控制。
    第三项不超过 $\varepsilon n \cdot c \cdot \xi(\ell)$,其中 $c$ 是一个常数,原因是重要性采样的权重之和不是n,c是一个由敏感度上界决定的常数$c=\frac{(\sum_{x_i\in X}s_i)/|X|}{(\sum_{x_i \in C}1/s_i)/|C|}$。
    因此,$|f(\theta, X) - f(\theta, C)| \leq \epsilon f(\theta, X) + (2+c) \epsilon n \xi(\ell)$。
    如果将 $\varepsilon$ 替换为 $\min \{\varepsilon / 2, \varepsilon \inf_{\theta \in \mathcal{B}(\tilde{\theta}, \epsilon)} f(\theta, X) / 2(2+c)n \xi(\ell)\}$,
    通过简单计算,我们可以以概率$1-\eta$得到大小为$|C|$的 $\epsilon$-coreset:
\begin{equation}
|C| = \Theta \left( \frac{S}{\varepsilon^2} \left( \log S+\text{ddim} \log \frac{1}{\epsilon} + \log \frac{1}{\eta} \right) \right)
\end{equation}
由于 $\xi(\ell)$ 有一个隐含参数 $\alpha$,$|C|$ 
的隐含常数取决于 $\alpha$,c 和
$\inf_{\theta \in \mathcal{B}(\tilde{\theta}, \epsilon)} \frac{f(\theta, X)}{n}$。
\begin{remark}
    在基于重要性采样的分析中,由于coreset的权重和不再是n,式中出现了常数c。实际上常数c的期望是1,因此其结论和之前一致。
\end{remark}

\end{proof}

\section{基于分层采样的coreset}

可能已经注意到,定理\ref{thm:importance sampling} (以及定理\ref{thm:improved importance sampling})中
提出的 coreset 大小是数据相关的。
即,coreset 大小取决于值 $S$,而对于不同的输入实例,$S$ 可能是不同的。
为了实现数据无关的 coreset 大小,我们引入了以下基于分层采样的方法。


我们设 $\varrho = \min_{x \in X} f(\tilde{\theta}, x)$ 和 $T = \frac{1}{|X|} f(\tilde{\theta}, X)$。
然后,我们根据取 $\theta=\tilde{\theta}$ 的损失函数将所有数据点划分到不同的层中。
具体来说,如果 $f(\tilde{\theta}, x) - \varrho < T$,则将点 $x$ 分配到第 0 层;
否则,我们将其分配到 $\left\lfloor \log \left( \frac{f(\tilde{\theta}, x) - \varrho}{T} \right) \right\rfloor$ 层。
令 $L$ 为落在第 $j$ 层中的点的集合 $X_j$ 的最大层数,$L$ 最多为 $\log n + 1$。
对于任何 $0 \leq j \leq L$,我们从每个 $X_j$ 中随机均匀地取一个小样本 $C_j$,
其中每个 $C_j$ 的点被赋予权重 $|X_j| / |C_j|$。最后,集合并 $\bigcup_{j=0}^{L} C_j$ 形成我们的最终 coreset。

\begin{theorem}
    对于一个CnB问题,其损失函数 $f(\theta, X)$ 如定义\ref{def:CnB} 所述,
    令 ddim 为参数空间的加倍维度。上述 coreset 构造方法
    可以在线性时间内实现一个大小为
    \begin{equation*}
    \Theta \left( \frac{\log n}{\epsilon^2} \left( \text{ddim} \cdot \log \frac{1}{\epsilon} + \log \frac{1}{\eta} \right) \right)
    \end{equation*}
    的 $\epsilon$-coreset。$|C|$的隐藏常数取决于 Lipschitz 常数 $\alpha$ 和 $\inf_{\theta \in \mathbb{B}(\tilde{\theta}, \ell)} \frac{1}{n} f(\theta, X)$。    
\end{theorem}

