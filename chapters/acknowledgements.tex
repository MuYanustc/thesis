% !TeX root = ../main.tex

\begin{acknowledgements}

行文至此,该写致谢了。毕业论文算是本科生活的一个句点,写着致谢的同时,也默默回忆自己的过去几年的经历,是该写点什么纪念一下。

我想,大学给我最好的礼物,是如何面对挫折。

来大学之前,我觉得自己一路顺风顺水,过关斩将,顺利的让人不可思议,我甚至也隐隐觉得
是不是太过顺利了一些。大一大二的生活风平浪静,虽然卷不了很高的绩点,
但也乐得滋味。嬉笑打闹的同学就在身边,偶尔遇见
一些值得一想的问题也能聊的尽兴,真是食髓知味啊,那样美好的日子。后面的几年里,生活突然就像翻了脸一样,追着我一顿锤。不管是生活还是学业,遇到了太多太多在当时的我
难以克服的问题。我也曾颓丧,也想放弃,躺在空无一人的宿舍里蹉跎时光,天天幻想着能否回到过去。但好在最后还是熬过来了。

现在想想,我终究还是幸运的。不管我是怎样的状态,我总能遇见一些好人,以很匪夷所思的角度拉我一把。
眼角微酸,我想,还是以正常的方式结束这段致谢,好好想想我应该感谢的人。

感谢我的父母,理解和尊重我,他们的支持和帮助是我最大的动力。感谢我的室友们,一起同窗度过这人生中灿烂的几年。
感谢我的各位朋友,他们如今大多漂落在五湖四海,但总能在我需要的时候提供各种各样的帮助。感谢身在上海的袁同学,他在我最苦难的一段日子里给了我很多陪伴和帮助。
感谢身在浙江的夏老师,她的想法总是能让我有所启发,和她交流总是能帮到我许多。
感谢不知道在哪飞的胡老师,感谢那些我生活中散发着光芒的小伙伴。
感谢身在深圳的老杨头,要谢你的地方实在太多了,认识快十年的老伙计了。
也有许多科大的老师想要感谢,他们也许有些不记得我,但他们给我的帮助是深刻而久远的。
感谢杨亚宁老师的精彩教学,帮我捡起了能学统计的信心;感谢杨青老师的善解人意,和她的几次交流都很愉快;
感谢梅玫老师和卢荣德老师,感谢他们的付出和指导;感谢我的导师丁虎老师,他的指导和帮助让我有能力完成这篇论文,他对待学术的态度
令我印象深刻。最后感谢我自己,感谢我在这几年里的坚持和努力,感谢我在这几年里的成长和收获。

希望自己,比昨天多一点,比明天少一点。



\end{acknowledgements}

